\section{Сравнительный анализ Linux, MacOS и Windows}

Главной отличительной чертой Linux системы является ее модульность и обширный набор системных программ, которые позволяли создать благоприятную обстановку для пользователей-программистов. Система UNIX органически сочетается с языком Си, на котором написано более 90\% ее собственных модулей. Командный язык системы практически совпадает с языком Си, что позволяло очень легко комбинировать различные программы при создании больших прикладных систем.

UNIX имеет <<оболочку>>, с которой пользователь непосредственно взаимодействует, и <<ядро>>, которое, собственно, и управляет действиями компьютера. Компьютер выводит в качестве приглашения для ввода команд долларовый знак. Из-за продолжительности пользования этой операционной системы количество команд весьма велико. В добавление к командам по управлению файлами, которые присутствуют в любой операционной системе, UNIX имеет, по крайней мере, один текстовый редактор, а также форматер текста и компилятор языка Си, что позволяет, по мере надобности, модифицировать <<оболочку>>.

От UNIX многие другие операционные системы переняли такие функции, как переназначение, канал и фильтр; однако UNIX имеет, несомненно, преимущество в том, что она с самого начала разрабатывалась как многопользовательская и многозадачная операционная система. Имена файлов могут иметь 14 знаков, причём в именах файлов различаются заглавные и строчные буквы. Первоначальный набор команд операционной системы расширился до 143 в версии 7.0; в версии System III добавилась ещё 71 команда, ещё 25 - в Berkeley 4.1 и следующие 114 в Berkeley 4.2. Из-за такого обилия команд UNIX не относится к самым удобным для пользователя языкам. Работа облегчается, если применять графический пользовательский интерфейс, но поскольку такое количество команд и без того занимает значительный объём памяти, этот интерфейс требует ещё большего объёма памяти и пространства диска.

С тем, что такое операционные системы и их особенностями в целом, мы разобрались, теперь самое время приступить к более детальному, конкретному рассмотрению многообразия ОС, которое обычно начинается с рассмотрения краткой истории появления и развития

Считается, что в появлении Юникса в частности виновата компьютерная игра. Дело в том, что Кен Томпсон непонятно чего ради создал игрушку <<Space Travel>>. Он написал ее в 1969 году на компьютере Honeywell-635, который использовался для разработки Multics. Но фишка в том, что ни вышеупомянутый Honeywell, ни имевшийся в лаборатории General Electric-645 не подходили для игрушки. И Кену пришлось найти другую ЭВМку - 18-разрядный компьютер РDР-7. Кен с ребятами разрабатывал новую файловую систему, дабы облегчить себе жизнь и работу. Ну и решил опробовать свое изобретение на новенькой машине. Опробовал. Весь отдел патентов Bell Labs дружно радовался. Томпсону этого показалось мало и он начал ее усовершенствовать, включив такие функции как inodes, подсистему управления процессами и памятью, обеспечивающую использование системы двумя пользователями в режиме TimeSharing'а (разделения времени) и простой командный интерпретатор. Кен даже разработал несколько утилит под систему. Собственно, сотрудники Кена еще помнили, как они мучались над ОС Multics, поэтому в честь старых заслуг один из них - Брайан Керниган - решил назвать ее похожим именем - UNICS. Через некоторое время название сократили до UNIX (читается так же, просто писать лишнюю букву настоящим программистам во все времена было лень). ОС была написана на ассемблере. Вот мы и подбираемся к тому, что известно в мире как «Первая редакция UNIX». В ноябре 1971 года был опубликован первый выпуск полноценной доки по Юниксу. В соответствии с этим и ОС была названа «Первой редакцией UNIX». Вторая редакция вышла довольно быстро - меньше, чем через год. Третья редакция ничем особенным не отличалась. Разве что заставила Дениса Ритчи «засесть за словари», вследствие чего тот написал собственный язык, известный сейчас как С. Именно на нём была написана 4-я редакция UNIX в 1973 году. В июле 1974 года вышла 5-я версия UNIX. Шестая редакция UNIX (аkа UNIX V6), выпущенная в 1975 году, стала первым коммерчески распространяемым Юниксом. Большая ее часть была написана на С.

Позже была полностью переписана подсистема управления оперативной и виртуальной памятью, заодно изменили интерфейс драйверов внешних устройств. Все это позволило сделать систему легко переносимой на другие архитектуры и было названо «Седьмая редакция» (aka UNIX version 7). Когда в 1976 году в Университет Беркли попала «шестерка», там возникли местные Юникс-гуру. Одним из них был Билл Джой.

Собрав своих друзей-программистов, Билли начал разработку собственной системы на ядре UNIX .Запихнув помимо основных функций кучу своих (включая, компилятор Паскаля), он назвал всю эту сборную солянку Distribution (BSD 1.0). Вторая версия BSD почти ни чем не отличалась от первой. Третья версия BSD основывалась на переносе UNIX Version 7 на компьютеры семейства VAX, что дало систему 32/V, легшую в основу BSD 3.x. Ну, и самое главное - при этом был разработан стек протоколов ТСР/IР; разработка финансировалась Министерством Безопасности США.

Первая коммерческая система называлась UNIX SYSTEM III и вышла она в 1982 году. В этой ОС сочетались лучшие качества UNIX Version 7.Далее Юниксы развивались примерно так: во-первых, появились компании, занимавшиеся коммерческим переносом UNIX на другие платформы. К этому приложила руку и небезызвестная Microsoft Corporation, совместно с Santa Cruz Operation произведшая на свет UNIX-вариацию под названием XENIX, во-вторых, Bell Labs создала группу по развитию Юникса и объявила о том, что все последующие коммерческие версии UNIX (начиная с System V) будут совместимы с предыдущими.

К 1984-му году был выпущен второй релиз UNIX System V, в котором появились: возможности блокировок файлов и записей, копирования совместно используемых страниц оперативной памяти при попытке записи (сору-on-write), страничного замещения оперативной памяти и т. д. К этому времени ОС UNIX была установлена на более чем 100 тыс. компьютеров.

В 1987-м году выпущен третий релиз UNIX System V. Было зарегистрировано четыре с половиной миллиона пользователей этой эпической операционной системы...Кстати, что касается Linux’а, то он возник лишь в 1990 году, а первая официальная версия ОС вышла лишь в октябре 1991 . Как и BSD, Linux распространялся с исходниками, чтобы любой пользователь мог настроить ее себе так, как ему хочется. Настраивалось практически ВСЕ, чего не может себе позволить, например, Windows 9x.

% MacOS

Разработчиком MacOS является фирма Apple - законодатель мод в области GUI, начиная с 1980-х гг. Ключевой идеей MacOS с самого начала является разработка и развитие ОС только на основе графического пользовательского интерфейса - "ОС без командной строки". Аппаратная платформа MacOS – всевозможные семейства компьютеров Macintosh фирмы Apple (наиболее популярные среди рабочих станций в США), а также PowerPC – рабочая станция RISC-архитектуры, совместно разработанная Apple, IBM и HP. Диалекты (версии) MacOS различаются по своему подходу к реализации, хотя для пользователя, благодаря удобному графическому интерфейсу, эти различия могут быть незаметны. Класическая MacOS (classic MacOS) - оригинальная разработка фирмы Apple; новая линия MacOS X – развитие ОС MacOS Classic и ОС NeXTSTEP (UNIX-подобной ОС), т.е. она является UNIX-совместимой.

Для компьютеров Macintosh характерен способ соединения устройств Apple Desktop bus (сокращённо ADB) – специальный вид соединений для подключения мыши, клавиатуры и джойстика к компьютеру. При его использовании все устройства могут соединятся между собой, и только одно следует подключить к машине непосредственно. Такой способ подключения называется цепочкой. Также возможно последовательное соединение (Daisy chaning) ряда дополнительных устройств, таких как проигрыватель компакт-дисков, жёсткий диск и сканер.

Macintosh-машины имеют встроенную поддержку звука и возможность подключения к сети, а также обеспечивают функцию «включи и работай» (plug and play), т.е. при подключении нового оборудования необходимо только произвести его наладку и согласовать с имеющимся. Компьютеры Macintosh Power или Macintosh AV при помощи адаптера Geo Port соединяются непосредственно с телефонной линией, позволяя связываться с оперативными службами, получать и передавать факсы без использования модема.

Персональные компьютеры Macintosh представлены несколькими видами. Компактные (моноблочные) модели целиком находятся в одном корпусе. Они могут иметь встроенный или внешний винчестер, у них четкие, но малые по размеру экраны, не отображающие цвета или полутона, мало оперативной памяти и нет гнезд для плат расширения. Модульные Macintosh, в отличии от компактных, состоят из нескольких частей или модулей. Все процессорные блоки у таких моделей имеют одно или несколько гнезд расширения (слотов). Мобильные (портативные) Macintosh типа Power book представляют собой миниатюрные, легко транспортируемые модели. Небольшие размеры машин этой серии, однако, ограничивают возможности и могут способствовать медленной работе. Вместо мышки (mouse) мобильные модели имеют встроенные шаровые манипуляторы – колобки (trackball) или чувствительную к перемещению пальца площадку – сенсорный планшет или пятачок (trackpad), что позволяет экономить место.

С момента выпуска первого компьютера Macintosh в 1984 году на всех машинах этого семейства и совмещенных с ними используется операционная система Mac Os, которая, конечно, может быть дополнена. Параметры её варьируются от версии к версии. Для функционирования компьютера наиболее необходимой является «Системная Папка» (System Folder), которая содержит всю основную информацию, задающую систему Mac Os. В ней находятся System file, Finder, шрифты, реквизиты (прикладные мини-программы, а также другие часто используемые программы), принтерные шрифты и т.д., она всегда выделяется особой пиктограммой. Из неё можно легко установить необходимые программы.


% Windows

Windows была, наверное, первой операционной системой, которую Биллу Гейтсу никто не заказывал, а разрабатывать ее он взялся на свой страх и риск. Что в ней такого особенного? Во-первых, графический интерфейс. Такой на тот момент был только у пресловутой Мас 0S. Во-вторых, многозадачность. В общем, в ноябре 1985 вышла Windows 1.0. Основной платформой ставились 286-е машины.

Ровно через два года, в ноябре 87-го вышла Windows 2.0, через полтора года вышла 2.10. Ничего особенного в них не было. И вот, наконец, революция! Май 1990-го года, вышла Windows 3.0. Чего там только не было: и ДОС-приложения выполнялись в отдельном окне на полном экране, и Сору-Paste работал для обмена с данными ДОС - приложений, и сами Винды работали в нескольких режимах памяти: в реальном (базовая 640 Кб), в защищенном и расширенном. При этом можно было запускать приложения, размер которых превышает размер физической памяти. Имел место быть и динамический обмен данными (DDE). Через пару лет вышла и версия 3.1, в которой уже отсутствовали проблемы с базовой памятью. Также была введена новомодная функция, поддерживающая шрифты True Туре. Обеспечена нормальная работа в локальной сети. Появился Drag\&Drop (перенос мышкой файлов и директорий). В версии 3.11 была улучшена поддержка сети и введено еще несколько малозначительных функций. Параллельно вышла Windows NT 3.5, которая была на тот момент сбором основных сетевых примочек, взятых из 0S/2.В июне 1995 вся компьютерная общественность была взбудоражена сообщением Microsoft о релизе в августе новой операционной системы, существенно иной, нежели Windows 3.11.

24 августа - дата официального релиза Windows 95 (другие названия: Windows 4.0, Windows Chicago).Теперь это была не просто операционная среда - это была полноценная операционная система. 32-битное ядро позволяло улучшить доступ к файлам и сетевым функциям. 32-битные приложения были лучше защищены от ошибок друг друга, имелась и поддержка многопользовательского режима на одном компьютере с одной системой. Множество отличий в интерфейсе, куча настроек и улучшений.

Чуть позже вышла новая Windows NT с тем же интерфейсом, что и 95-е. Поставлялась в двух вариантах: как сервер и как рабочая станция. Системы Windows NТ 4.x были надежны, но не столько потому, что у Microsoft проснулась совесть, сколько потому, что NТ писали программисты, когда-то работавшие над VАХ/VMS.

