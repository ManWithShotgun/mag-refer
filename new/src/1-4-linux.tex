\section{Семейство Linux систем}

К операционной системе GNU/Linux также часто относят программы, дополняющие эту операционную систему, и прикладные программы, делающие её полноценной многофункциональной операционной средой. В отличие от большинства других операционных систем, GNU/Linux не имеет единой «официальной» комплектации. Вместо этого GNU/Linux поставляется в большом количестве так называемых дистрибутивов, в которых программы GNU соединяются с ядром Linux и другими программами.

В отличие от Microsoft Windows, Mac OS и коммерческих UNIX-подобных систем, GNU/Linux не имеет географического центра разработки. Нет и организации, которая владела бы этой системой; нет даже единого координационного центра. Программы для Linux — результат работы тысяч проектов. Некоторые из этих проектов централизованы, некоторые сосредоточены в фирмах. Многие проекты объединяют хакеров со всего света, которые знакомы только по переписке. Создать свой проект или присоединиться к уже существующему может любой и, в случае успеха, результаты работы станут известны миллионам пользователей. Пользователи принимают участие в тестировании свободных программ, общаются с разработчиками напрямую, что позволяет быстро находить и исправлять ошибки и реализовывать новые возможности.

Именно такая гибкая и динамичная система разработки, невозможная для проектов с закрытым кодом, определяет исключительную экономическую эффективность GNU/Linux. Низкая стоимость свободных разработок, отлаженные механизмы тестирования и распространения, привлечение людей из разных стран, обладающих разным видением проблем, защита кода лицензией GPL — всё это стало причиной успеха свободных программ.

Конечно, такая высокая эффективность разработки не могла не заинтересовать крупные фирмы, которые стали открывать свои проекты. Так появились Mozilla (Netscape, AOL), OpenOffice.org (Sun), свободный клон Interbase (Borland) — Firebird, SAP DB (SAP). IBM способствовала переносу GNU/Linux на свои мейнфреймы.

С другой стороны, открытый код значительно снижает себестоимость разработки закрытых систем для GNU/Linux и позволяет снизить цену решения для пользователя. Вот почему GNU/Linux стала платформой, часто рекомендуемой для таких продуктов, как Oracle, DB2, Informix, SyBase, SAP R3, Domino.

Большинство пользователей для установки GNU/Linux используют дистрибутивы. Дистрибутив — это не просто набор программ, а ряд решений для разных задач пользователей, объединённых едиными системами установки, управления и обновления пакетов, настройки и поддержки.~\cite{linuxOffDoc}

Самые распространённые в мире дистрибутивы:
\begin{enumerate}

\item Ubuntu. Быстро завоевавший популярность дистрибутив, ориентированный на лёгкость в освоении и использовании.

\item openSUSE. Бесплатно распространяемая версия дистрибутива SuSE, принадлежащая компании Novell. Отличается удобством в настройке и обслуживании благодаря использованию утилиты YaST.
Fedora. Поддерживается сообществом и корпорацией RedHat, предшествует выпускам коммерческой версии RHEL.

\item Debian. Международный дистрибутив, разрабатываемый обширным сообществом разработчиков в некоммерческих целях. Послужил основой для создания множества других дистрибутивов. Отличается строгим подходом к включению несвободного ПО.
\item Mandriva. Французско-бразильский дистрибутив, объединение бывших Mandrake и Conectiva.
Slackware. Один из старейших дистрибутивов, отличается консервативным подходом в разработке и использовании.

\item Gentoo. Дистрибутив, собираемый из исходных кодов. Позволяет очень гибко настраивать конечную систему и оптимизировать производительность, поэтому часто называет себя мета-дистрибутивом. Ориентирован на экспертов и опытных пользователей.

\item Archlinux. Ориентированный на применение самых последних версий программ и постоянно обновляемый, поддерживающий одинаково как бинарную, так и установку из исходных кодов и построенный на философии простоты «KISS» («Keep it simple, stupid» / «Не усложняй»), этот дистрибутив ориентирован на компетентных пользователей, которые хотят иметь всю силу и модифицируемость Linux, но не в жертву времени обслуживания.
\end{enumerate}

Помимо перечисленных, существует множество других дистрибутивов, как базирующихся на перечисленных, так и созданных с нуля и зачастую предназначенных для выполнения ограниченного количества задач.

Каждый из них имеет свою концепцию, свой набор пакетов, свои достоинства и недостатки. Ни один не может удовлетворить всех пользователей, а потому рядом с лидерами благополучно существуют другие фирмы и объединения программистов, предлагающие свои решения, свои дистрибутивы, свои услуги. Существует множество LiveCD, построенных на основе GNU/Linux, например, Knoppix. LiveCD позволяет запускать GNU/Linux непосредственно с компакт-диска, без установки на жёсткий диск. Большинство крупных дистрибутивов, включая Ubuntu, могут быть использованы как LiveCD.

Дистрибутивы Linux часто бывают ориентированы на конкретные задачи. Поэтому не получится просто составить список операционных систем и сказать: «они – самые лучшие». Здесь выделены несколько областей использования Linux и выбраны те дистрибутивы, у которых есть все шансы стать первыми в своей нише в 2017-м.~\cite{linuxDistr}

Лучший дистрибутив для системных администраторов это Parrot Linux. У любого администратора всегда полно работы. Без хорошего набора инструментов его дни – это постоянное испытание на прочность, непрерывная гонка. Однако, существует множество дистрибутивов Linux, готовых прийти на помощь. Один из них – Parrot Linux. Уверен, он приобретёт серьёзную популярность в 2017-м.

Этот дистрибутив основан на Debian, он предлагает огромное количество средств для испытания защищённости систем от несанкционированного доступа. Тут, кроме того, можно найти инструменты из сферы криптографии и компьютерной криминалистики, средства для работы с облачными службами и пакеты для обеспечения анонимности. Есть здесь и кое-что для разработчиков, и даже программы для организации времени. Всё это (на самом деле, там – море инструментов) работает на базе стабильной, проверенной временем системы. В результате получился дистрибутив, отлично подходящий для специалистов по информационной безопасности и сетевых администраторов.

Лучший легковесный дистрибутив LXLE.  LXLE сочетает в себе достойные возможности и скромные системные требования. Другими словами, это дистрибутив, который занимает мало места, но позволяет полноценно работать на компьютере. В LXLE можно найти всё необходимое, характерное для релизов Linux, рассчитанных на настольные ПК. Система вполне подойдёт для дома, под её управлением смогут работать не самые новые компьютеры (не говоря уже о вполне актуальных системах).

LXLE основана на Ubuntu 16.04 (как результат – долговременная техподдержка обеспечена), здесь применяется менеджер рабочего стола LXDE, который, за более чем десять лет существования, знаком многим, да и устроен несложно.

После установки LXDE у вас под рукой окажется множество стандартных средств, вроде LibreOffice и Gimp. Единственно, надо будет самостоятельно установить современный браузер.

Лучший корпоративный серверный дистрибутив RHEL. По данным Gartner~\cite{linuxOffDoc}, RHEL принадлежит 67\% рынка Linux-дистрибутивов для крупных организаций, при этом подписка на RHEL приносит компании Red Hat около 75\% доходов. У такого положения дел много причин. Так, Red Hat предлагает корпоративным клиентам именно то, что им нужно, но, кроме этого, компания прикладывает огромные усилия к развитию множества проектов с открытым исходным кодом.

Red Hat знает – что такое Linux, и что такое – корпоративный сектор. Red Hat доверяет немало компаний из списка Fortune 500 (например, ING, Sprint, Bayer Business Services, Atos, Amadeus, Etrade). Дистрибутив RHEL вывел множество разработчиков на новый уровень в областях безопасности, интеграции, управления, в сфере работы с облачными системами.


