\section{Основные операционные системы}

В наше время существует огромное множество типов операционныхсистем, имеющих различные области применения. В таких условиях можновыделить четыре основных критерия, описывающих назначение ОС.

Операционная система (ОС) — комплекс взаимосвязанных программ,предназначенных   для   управления   ресурсами   вычислительного   устройства.Благодаря   этим   программам   происходит   организация   взаимодействия   спользователем.  Управление   памятью,   процессами,   и   всем   программным   иаппаратным обеспечением устраняет необходимость работы непосредственно сдисками и предоставляет простой, ориентированный на работу с файламиинтерфейс,   скрывает   множество   неприятной   работы   с   прерываниями,счетчиками времени, организацией памяти и другими компонентами.~\cite{Oc1}

Организация   удобного   интерфейса,   позволяющая   пользователювзаимодействовать   с  аппаратурой   компьютера   за   счет   некой   расширеннойвиртуальной   машины,   с   которой   удобнее   работать   и   которую   легчепрограммировать.   Вот   перечень   основных   сервисов,   предоставляемыхтипичными операционными системами.Разработка программ, где ОС представляет программисту разнообразныеинструменты разработки приложений: редакторы, отладчики и т.п. Ему необязательно   знать,   как   функционируют   различные   электронные   иэлектромеханические узлы и устройства компьютера. Часто пользователь можетобойтись   только   мощными   высокоуровневыми   функциями,   которыепредставляет ОС.Также, для запуска программы нужно выполнить ряд действий: загрузитьв основную память программу и данные, инициализировать устройства ввода-вывода и файлы, подготовить другие ресурсы. ОС выполняет всю эту работувместо пользователя. ОС дает доступ к устройствам ввода-вывода. Каждое устройство требуетсвой набор команд для запуска. ОС предоставляет пользователю единообразныйинтерфейс,   который   опускает   все   детали   и   дает   программисту   доступ   кустройствам ввода-вывода через простейшие команды чтения и записи. При работе с файлами управление со стороны ОС предполагает не толькоглубокий учет природы устройства ввода-вывода, но и знание структур данных,записанных в файлах. Многопользовательские ОС, кроме того, обеспечиваютмеханизм защиты при обращении к файлам. ОС управляет доступом к совместно используемой или общедоступнойвычислительной системе в целом, а также к отдельным системным ресурсам.Она   обеспечивает   защиту   ресурсов   и   данных   от   несанкционированногоиспользования и разрешает конфликтные ситуации.

Обнаружение ошибок и их обработка— это еще один очень важныймомент   в   назначении   ОС.   При   работе   компьютерной   системы   могутпроисходить разнообразные сбои за счет внутренних и внешних ошибок ваппаратном   обеспечении,   различного   рода   программных   ошибок(переполнение,   попытка   обращения   к   ячейке   памяти,   доступ   к   которойзапрещен и др.). В каждом случае ОС выполняет действия, минимизирующиевлияние ошибки на работу приложения (от простого сообщения об ошибке доаварийной остановки программы).И, наконец, учет использования ресурсов. ОС  имеет средства учетаиспользования   различных   ресурсов   и   отображения   параметровпроизводительности  вычислительной системы. Эта информация важна длянастройки (оптимизации) вычислительной системы с целью повышения еепроизводительности.~\cite{Oc1}

Организация   эффективного   использования   ресурсов   компьютера.   ОСтакже является своеобразным диспетчером ресурсов компьютера. К числуосновных ресурсов современных вычислительных систем относятся основнаяпамять, процессоры,   таймеры, наборы данных, диски, накопители на МЛ,принтеры, сетевые устройства, и др. Перечисленные ресурсы определяютсяоперационной системой между выполняемыми программами. В отличие отпрограммы, которая является статическим объектом, выполняемая программа –это динамический объект, который называется процессом и является базовымпонятием современных ОС. Управление ресурсами вычислительной системы сцелью наиболее эффективного их использования является вторым назначениемоперационной системы. Критерии эффективности, в соответствии с которымиОС организует управление ресурсами компьютера, могут быть различными.Например, в одном случае наиболее важным является пропускная способностьвычислительной систем, в другом – время ее реакции. Зачастую ОС должныудовлетворять   нескольким,   противоречащим   друг   другу   критериям,   чтодоставляет   разработчикам   серьезные   трудности.   Управление   ресурсамивключает решение ряда общих, не зависящих от типа ресурса задач.Планирование ресурса – определение процесса, для которого необходимовыделить ресурс. Здесь предопределяется, когда и в каком качестве долженвыделиться данный ресурс. Удовлетворение запросов на ресурсы – выделениересурсов процессам; мониторинг состояния и учет использования ресурса –поддержание   оперативной   информации   о   задействовании   ресурса   ииспользовании   его   доли.   Разрешение   конфликтов   между   процессами,претендующими на один и тот же ресурс. Для   решения   этих   общих   задач   управления   ресурсами   разные   ОСиспользуют различные алгоритмы, что в итоге и определяет облик ОС в целом,включая   характеристики   производительности,   область   применения   и   дажепользовательский интерфейс.~\cite{Oc1}

Облегчение процессов эксплуатации аппаратных и программных средстввычислительной системы. Ряд операционных систем имеет в своем составенаборы   служебных   программ,   обеспечивающие   резервное   копирование,архивацию данных, проверку, очистку и дефрагментацию дисковых устройств идр. Кроме того, современные ОС имеют достаточно большой набор средств испособов диагностики и восстановления работоспособности системы. Сюдаотносятся: -диагностические программы для выявления ошибок в конфигурацииоперационной системы; -средства восстановления последней работоспособной конфигурации; -средства   восстановления   поврежденных   и   пропавших   системныхфайлов и др.~\cite{Oc1}

Современные   ОС   организуются   таким   образом,   что   допускаютэффективную   разработку,   тестирование   и   внедрение   новых   системныхфункций,   не   прерывая   процесса   нормального   функционированиявычислительной   системы.   Большинство   операционных   систем   постоянноразвиваются (нагляден пример Windows). Происходит это в силу следующихпричин. [1]Для удовлетворения пользователей или нужд системных администраторовОС должны постоянно предоставлять новые возможности. Например, можетпотребоваться   добавить   новые   инструменты   для   контроля   или   оценкипроизводительности,   новые   средства   ввода-вывода   данных   (речевой   ввод).Другой пример – поддержка новых приложений, использующих окна на экранедисплея. [1]В каждой ОС есть ошибки. Время от времени они обнаруживаются иисправляются. Отсюда постоянные появления новых версий и редакций ОС.Необходимость регулярных изменений накладывает определенные требованияна организацию операционных систем. Очевидно, что эти системы должныиметь модульную структуру с четко определенными межмодульными связями.Важную роль играет хорошая и полная документированность системы.~\cite{Oc1}

Функции   ОС   обычно   группируются   либо   в   соответствии   с   типамилокальных   ресурсов,   которыми   управляет   ОС,   либо   в   соответствии   соспецифическими задачами, применимыми ко всем ресурсам. Совокупностимодулей,   выполняющих   такие   группы   функций,   образуют   подсистемыоперационной   системы.   Наиболее   важными   подсистемами   управленияресурсами являются подсистемы управления процессами, памятью, файлами ивнешними устройствами, а подсистемами, общими для всех ресурсов, являютсяподсистемы   пользовательского   интерфейса,   защиты   данных   иадминистрирования.~\cite{Oc2}

Подсистема   управления   процессами   непосредственно   влияет   нафункционирование   вычислительной   системы.   Для   каждой   выполняемойпрограммы ОС организует один или более процессов. Каждый такой процесспредставляется в ОС информационной структурой (таблицей, дескриптором,контекстом   процессора),   содержащей   данные   о   потребностях   процесса   вресурсах, а также о фактически выделенных ему ресурсах (область оперативнойпамяти, количество процессорного времени, файлы, устройства ввода-вывода идр.).   В   современных   мультипрограммных   ОС   может   существоватьодновременно   несколько   процессов,   порожденных   по   инициативепользователей и их приложений, а также инициированных ОС для выполнениясвоих   функций   (системные   процессы).   Поскольку   процессы   могутодновременно претендовать на одни и те же ресурсы, подсистема управленияпроцессами планирует очередность выполнения процессов, обеспечивает ихнеобходимыми   ресурсами,   обеспечивает   взаимодействие   и   синхронизациюпроцессов.~\cite{Oc2}

Подсистема управления памятью производит распределение физическойпамяти   между   всеми   существующими   в   системе   процессами,   загрузку   иудаление программных кодов и данных процессов в отведенные им областипамяти,   а   также   защиту   областей   памяти   каждого   процесса.   Стратегияуправления   памятью   складывается   из   стратегий   выборки,   размещения   изамещения блока программы или данных в основной памяти. Соответственноиспользуются различные алгоритмы, определяющие, когда загрузить очереднойблок в память, в какое место памяти его поместить и какой блок программы илиданных удалить из основной памяти, чтобы освободить место для размещенияновых блоков. Одним из наиболее популярных способов управления памятью всовременных   ОС   является   виртуальная   память.   Реализация   механизмавиртуальной памяти позволяет программисту считать, что в его распоряженииимеется однородная оперативная память, объем которой ограничивается тольковозможностями адресации, предоставляемыми системой программирования.

Нарушения защиты памяти связаны с обращениями процессов к участкампамяти, выделенной другим процессам прикладных программ или программсамой ОС. Средства защиты памяти должны пресекать такие попытки доступапутем аварийного завершения программы-нарушителя.

Функции управления файлами сосредоточены в файловой системе ОС.Операционная система виртуализирует отдельный набор данных, хранящихсяна   внешнем   накопителе,   в   виде   файла   –   простой   неструктурированной последовательности байтов, имеющих символьное имя. Для удобства работы сданными файлы группируются в каталоги, которые, в свою очередь, образуютгруппы – каталоги более высокого уровня. Файловая система преобразуетсимвольные имена файлов, с которыми работает пользователь или программист,в физические адреса данных на дисках, организует совместный доступ кфайлам, защищает их от несанкционированного доступа.  [4]

Функции управления внешними устройствами возлагаются на подсистемууправления внешними устройствами, называемую также подсистемой ввода-вывода.   Она   является   интерфейсом   между   ядром   компьютера   и   всемиподключенными к нему устройствами. Спектр этих устройств очень обширен(принтеры, сканеры, мониторы, модемы, манипуляторы, сетевые адаптеры,АЦП разного рода и др.), сотни моделей этих устройств отличаются набором ипоследовательностью   команд,   используемых   для   обмена   информацией   спроцессором   и   другими   деталями.   Программа,   управляющая   конкретноймоделью внешнего устройства и учитывающая все его особенности, называетсядрайвером. Наличие большого количества подходящих драйверов во многомопределяет   успех   ОС   на   рынке.   Созданием   драйверов   занимаются   какразработчики ОС, так и компании, выпускающие внешние устройства. ОСдолжна поддерживать четко определенный интерфейс между драйверами иостальными   частями   ОС.   Тогда   разработчики   компаний-производителейустройств ввода-вывода могут поставлять вместе со своими устройствамидрайверы для конкретной операционной системы. [4]

Безопасность   данных   вычислительной   системы   обеспечиваетсясредствами отказоустойчивости ОС, направленными на защиту от сбоев иотказов аппаратуры и ошибок программного обеспечения, а также средствамизащиты от несанкционированного доступа. Для каждого пользователя системыобязательна процедура логического входа, в процессе которой ОС убеждается,что в систему входит пользователь, разрешенный административной службой.Корпорация  Microsoft, например, в своем последнем продукте  Windows  10предлагает пользователю вход в систему через распознавание внешности. Этодолжно повысить безопасность и сделать вход в систему быстрее. [6]А вот  Google  обещает нам в новой версии своей ОС для смартфоновAndroid  6.0  доступ   к   устройству  и   подтверждение   покупок   через   сканеротпечатка пальца, если для того пригодно устройство. [7]Администратор   вычислительной   системы   определяет   и   ограничиваетвозможности   пользователей   в   выполнении   тех   или   иных   действий,   т.е.определяет   их   права   по   обращению   и   использованию   ресурсов   системы.Важным средством защиты являются функции аудита ОС, заключающегося вфиксации всех событий, от которых зависит безопасность системы. Поддержкаотказоустойчивости   вычислительной   системы   реализуется   на   основе резервирования   (дисковые   RAID-массивы,   резервные   принтеры   и   другиеустройства, иногда резервирование центральных процессоров, в ранних ОС –дуальные и дуплексные системы, системы с мажоритарным органом и др.).Вообще   обеспечение   отказоустойчивости   системы   –   одна   из   важнейшихобязанностей системного администратора, который для этого использует рядспециальных средств и инструментов.~\cite{Oc2}

Прикладные программисты используют в своих приложениях обращенияк операционной системе, когда для выполнения тех или иных действий имтребуется   особый   статус,   которым   обладает   только   ОС.   Возможностиоперационной   системы   доступны   программисту   в   виде   набора   функций,который называется интерфейсом прикладного программирования (ApplicationProgramming   Interface,  API).   Приложения   обращаются   к   функциям   API   спомощью системных вызовов. Способ, которым приложение получает услугиоперационной   системы,   очень   похож   на   вызов   подпрограмм.   Способреализации   системных   вызовов   зависит   от   структурной   организации   ОС,особенностей аппаратной платформы и языка программирования. В ОС UNIXсистемные вызовы почти идентичны библиотечным процедурам.~\cite{Oc1}

ОС   обеспечивает   удобный   интерфейс   не   только   для   прикладныхпрограмм,   но   и   для   пользователя   (программиста,   администратора,пользователя). На данный момент производители предлагают нам множествофункций, призванных облегчить нашу работу с устройствами и сэкономитьвремя. В качестве примера я опять хочу привести  Windows  10.  Microsoftпомогает   пользователю   обеспечить   беспрепятственную   работу   всех   егоустройств (естественно от  Microsoft) , за счет общей ОС. Тут и мгновеннаяпередача данных с одного устройства на другое, и общие уведомления, которыес такой функцией никак не пропустишь. [6]<<Эффективная, организованная работа>>  – это практически слоган длякаждого производителя ОС. Работа с заметками прямо на веб-страницах, новыемногооконные режимы, несколько рабочих столов – все это мы видим уже какнесколько лет, а у разработчиков еще много идей. [6]

Современные   операционные   системы   имеют   сложную   структуру,состоящую   из   множества   элементов,   где   каждый   из   них   выполняетопределенные функции по управлению процессами и распределению ресурсов.
Ядро ОС – центральная часть операционной системы, обеспечивающаяприложениям   координированный   доступ   к   файловой   системе,   и   обменуфайлами между ПУ. [4]

Командный процессор. Программный модуль ОС, ответственный за чтение отдельных командили же последовательности команд из командного файла, иногда называюткомандным интерпретатором. [4]

Драйверы устройств. К магистрали  компьютера подключаются различные устройства(дисководы,   монитор,   клавиатура,   мышь,   принтер   и   др.).   Каждоеустройство выполняет определенную функцию, при этом техническаяреализация устройств существенно различается. В состав операционнойсистемы входятдрайверы устройств, специальные программы, которыеобеспечивают   управление   работой   устройств   и   согласованиеинформационного обмена с другими устройствами, а также позволяютпроизводить   настройку   некоторых   параметров   устройств.   Каждомуустройству соответствует свой драйвер.[4]

Утилиты. Дополнительные сервисные программы (утилиты) – вспомогательныекомпьютерные   программы   в   составе   общего   программного   обеспечения,делающие   удобным   и   многосторонним   процесс   общения   пользователя   скомпьютером. [4]

Для   удобства   пользователя   в   состав   операционной   системы   обычновходит такжесправочная система. Справочная система позволяет оперативнополучить необходимую информацию как о функционировании операционнойсистемы в целом, так и о работе ее отдельных модулей. [4]


