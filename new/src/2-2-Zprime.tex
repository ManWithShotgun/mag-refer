\section{Инструменты имитационного моделирования}

Для исследования отклика детектора на различные физические процессы, созданы программы, позволяющие перевести моделированное на уровне частиц событие  взаимодействия протонов при соударении в формат представления данных детекторов установки ATLAS. Алгоритмы моделирования интегрированы в программную оболочку эксперимента  ATLAS, именуемую Athena, использующую программный пакет GEANT4.

Генератор события создает набор частиц, который направляется в программу быстрого или полного моделирования детектора. Генераторы событий встроены в Athena. Используется большое число других, поддерживаемых авторами, генераторов, которые имеют блоки связи для использования в Athena. Основной массив модельных событий создан с помощью генераторов PYTHIA~\cite{2part-pythia-all}, включая его версию PYTHIAВ,  предназначенную в  ATLAS для моделирования событий с рождением В-адронов.

PYTHIA - это программного пакета для визуализации результатов моделирования процессов столкновения частиц при высоких энергиях осуществляющего генерацию методом Монте-Карло физических событий.

Генераторы общего назначения создают событие как целое. Они используют много параметров, часть из которых относится к фундаментальным параметрам, такие как константы связи квантовой хромодинамики (КХД) и электрослабой теории, часть относится к моделям, описывающим взаимодействия на больших расстояниях, с малыми передачами импульса, т.н. «мягкой» КХД, и к электрослабым процессам.

Для корректного моделирования процессов рождения и распада частиц необходимо учитывать условия проведения эксперимента. Это условия рождения изучаемых частиц на ускорителе при соответствующих энергиях сталкивающихся пучков, полные цепочки распадов частиц до уровня "стабильных частиц", регистрируемых детектором. Для решения этих задач применяются генераторы событий, использующие метод Монте-Карло.
Генератор PYTHIA является широко используемой в физике высоких энергий программой моделирования столкновений различных частиц в широком диапазоне энергий. Этот генератор учитывает процессы фрагментации кварков в адроны и разыгрывает сложные цепочки адронных распадов. Стартуя с заданного пользователем процесса, (столкновение двух протонов с рождением Z-бозона и т.п.) программа случайным образом (с учетом законов сохранения и, по возможности, теоретически известной структуры взаимодействия) разыгрывает конфигурацию конечных партонов, а затем моделирует т.н. процесс адронизации - процесс превращения ненаблюдаемых кварков и глюонов в реальные стабильные и нестабильные частицы с последующим распадом нестабильных частиц. На выходе программа выдает список всех частиц, родившихся в результате столкновения заданных первичных частиц, значения их компонент импульса и энергии. Кроме того, имеется возможность проследить последовательность рождений и распадов от первичного взаимодействия до рождения данной частицы. В качестве входных параметров программы используются описания сталкивающихся частиц, их энергий и тип моделируемого процесса (например, рождение Z-бозона). Существующие версии пакета РYTHIА написаны на языке программирования FORTRAN.
Результаты генерации - характеристики вторичных частиц -записываются в файл, что позволяет в дальнейшем проводить статистическую обработку событий.