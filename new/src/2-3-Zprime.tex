\section{Распады промежуточных бозонов. Определение массы Z-бозона}

В физических программаx экспериментов на  современных  дронных (\textit{LHC}) и планируемых на  электрон-позитронных (\textit{ILC, CLIC}) коллайдерах вопросу поиск  <<новой>> физики, выходящей за  рамки Стандартной модели (СМ), традиционно уделяется большое внимание. К числу подобных теоретических построений, являющихся обобщением СМ, относятся модели с расширенным к либровочным сектором, такие как лево-правосиммет- 
ричные модели (\textit{LR}), альтернативные лево-правосимметричные модели (\textit{ALR}), $E_6$-модели
и др.~\cite{Bobovnikov:2016}. Их исследование (теоретическое и экспериментальное) представляет значи-тельный интерес. Эти модели являются одними из простейших расширений СМ, характеризующихся элементарной структурой хиггсовского сектора. Общим для данных моделей является то, что они предсказывают новые физические объекты и явления на масштабе энергий $O$ (1 ТэВ), связанные, например, с наличием тяжелых нейтральных ($Z^\prime$) калибровочных бозонов, обусловленных дополнительными калибровочными симме-триями $U(1)^\prime$.

Достижение порога рождения $Z^\prime$-бозона явилось бы прямым доказательством про-явления «новой» физики. Однако в данном случае интервал поиска масс $Z^\prime$ ограничен максимальной энергией коллайдера, на котором проводятся эксперименты. Значительно более широкий интервал масс можно исследовать с помощью пропагаторных эффектов. В этом случае ведется поиск отклонений различных наблюдаемых от соответствующих предсказаний СМ. Если экспериментальные данные при достигнутом уровне точности согласуются с СМ, т. е. отклонений от предсказаний СМ нет, то эту экспериментальную информацию можно использовать для получения ограничений на динамические параметры и массы $Z^\prime$-бозонов.

Потенциальные возможности $e^+$$e^-$-коллайдеров для прямого рождения новых калибровочных бозонов гораздо скромнее по сравнению с адронными машинами из-за более низких энергий пучков. Кроме того, современные ограничения на массы $Z^\prime$-бозонов для большинства моделей превосходят планируемую энергию электрон-позитронного коллайдера \textit{ILC}, $\sqrt{s}<< M_{Z^\prime}$. Тем не менее основным достоинством этих машин является возможность проведения экспериментов по измерению наблюдаемых величин с высокой степенью точности и получения однозначной информации о косвенных (виртуальных) эффектах новых $Z^\prime$-бозонов, а также эффектах бозонного $Z$-$Z^\prime$-смешивания. Последние, в моделях с расширенным калибровочным сектором, зависят от структуры хиггсовского сектора модели. Тем самым экспериментальное исследование процессов рождения пар $W^±$-бозонов может не только пролить свет на возможное существование <<новой>> физики, но и дать косвенные указания на хиггсовскую природу, а также установить структуру модели.

На основе данных, полученных из низкоэнергетических экспериментов по нейтральным токам, результатов на $e^+e^-$-коллайдерах \textit{LEP} и \textit{SLC}~\cite{Bobovnikov:2016}, а также недавно выполненных экспериментов по поиску прямого адронного рождения $Z^\prime$-бозонов в процессе Дрелла-Яна.\\
\begin{center}
	$pp \rightarrow Z^\prime \rightarrow l^+l^- + X$
\end{center}
($l=e,\mu$) на коллайдере \textit{LНC} при энергии $\sqrt{s}$ = 7 и 8 ТэВ с интегральной светимостью соответственно $L_int$ = 5 и 20 фб${}^{-1}$~\cite{Bobovnikov:2016} можно заключить, что для большинства расширенных калибровочных моделей граничные значения для масс дополнительных $Z^\prime$- бозонов находятся в интервале $\sim$ 2,5-3,0 ТэВ (в зависимости от модели), а современный масштаб ограничений на угол смешивания составляет $\mathcal{O}(\varphi )$ ~ ${10}^{-2}$-${10}^{-3}$ рад. При этом наиболее точная информация об угле смешивания была получена преимуще¬ственно из экспериментов на электрон-позитронных коллайдерах \textit{LEP1} и \textit{SLC} по измерению резонансных наблюдаемых физических величин при энергии начальных состояний, равной массе стандартного $Z$-бозона, $\sqrt{s}$ = $M_Z$, в процессах\\
\begin{center}
	$e^+e^- \rightarrow f\bar{f}$
\end{center}

где конечными фермионными состояниями $f$ были заряженные лептоны и кварки. Высокая точность, достигнутая в экспериментах на коллайдерах \textit{LEP1} и \textit{SLC}, объясняется прежде всего возможностью набора большого объема данных в резонансной области энергии.

Кроме того, эта информация дополнялась данными, полученными на коллайдере тэватрон, по точному измерению массы $M_W$, на основе которых определялся параметр бозонного $Z$−$Z^\prime$-смешивания с использованием соотношения между массами нейтральных и заряженных калибровочных бозонов, $M_Z$ = $M_W$/$(\sqrt{p_0}\cos\theta_W)$, имеющего место в расширенных моделях. Очевидно также, что эти данные будут дополнены новой информацией, которая в ближайшем будущем будет получена в экспериментах на коллайдере \textit{LНС} при энергии 13 и 14 ТэВ. Вместе с тем из этих данных нельзя сделать однозначный вывод о природе «новой» физики, который мог бы вызвать отклонение наблюдаемых величин от их поведения, предсказываемого СМ. Дело в том, что параметр $p$, который содержится в выражениях для векторных и аксиально-векторных констант связи фермионов с учетом петлевых поправок, зависит, в частности, от структуры хиггсовского сектора модели, которая изначально неизвестна. Кроме того, новые тяжелые фермионы и скалярные частицы, предсказываемые моделями с расширенным калибровочным сектором, могут давать вклад в параметр $p$ на петлевом уровне. Все эти неопределенности приводят к появлению систематических (теоретических) погрешностей, которые могут быть весьма существенными при измерении параметра $p$ и, в конечном счете, могут повлиять на точность определения параметра $Z$−$Z^\prime$-смешивания.

Процессы парного рождения заряженных $W^±$-бозонов в адронных столкновениях на \textit{LНС}
\begin{center}
	$pp \rightarrow W^+W^- + X$
\end{center}
электрон-позитронной аннигиляции на \textit{LЕР2} и в большей степени на \textit{ILС}
\begin{center}
	$e^+e^- \rightarrow W^+W^-$
\end{center}

Являются весьма эффективным инструментом поиска эффектов $Z$−$Z^\prime$-смешивания при высоких энергиях и, таким образом, играют роль основного поставщика информации об угле $Z$−$Z^\prime$-смешивания~\cite{Bobovnikov:2016}. С теоретической точки зрения процессы парного рождения заряженных калибровочных бозонов в адронных и электронпозитронных столкновениях интересны тем, что их сечения пропорциональны углу $Z$−$Z^\prime$-смешивания, который, как отмечалось выше, в расширенных калибровочных моделях зависит от структуры хиггсовского сектора.

Прямой поиск тяжелых резонансов в процессе $p\bar{p} \rightarrow W^+W^- + X$ осуществлялся экспериментальными группами \textit{СDF} и \textit{D0} на коллайдере тэватрон. Коллаборация \textit{D0} исследовала возможность рождения резонанса в канале его дибозонного распада, используя чисто лептонные $lvl^\prime v^\prime$ и полулептонные $vjj$ моды. Здесь $l=e,\mu$; $jj$ — две адронные струи. Коллаборация \textit{СDF} также осуществляла поиск тяжелых резонансов в канале их распада в пару заряженных калибровочных бозонов $W^+W^−$ с последующим распадом в полулептонные $evjj$ конечные состояния. Обе коллаборации установили ограничения на массы тяжелых резонансов, таких как новые нейтральные $Z^\prime$- и заряженные калибровочные $W^±$-бозоны, гравитоны Рэндалл-Сандрума. Кроме того, в настоящее время поиск тяжелых резонансов на \textit{LHC} в \textit{WW}-канале интенсивно ведется коллаборациями \textit{ATLAS} и \textit{CMS}. В частности, уже получена экспериментальная информация о процессе в лептонном канале $lvl^\prime v^\prime$ при энергии коллайдера 7 ТэВ и интегральной светимости 4,7 фб${}^{−1}$~\cite{2part-pankov}.

Из анализа экспериментальных данных по измерению процесса электрон-позитронной аннигиляции на коллайдере \textit{LEP2} были впервые получены прямые ограничения на угол $Z$−$Z^\prime$-смешивания. Точность измерения угла смешивания оказалась не очень высокой, $\left |\phi \right |$~5—10 \%, так как сам коллайдер работал в интервале энергий, незначительно превышающем порог реакции, $\sqrt{s} >> 2M_W$. Как было установлено ранее, чувствительность процесса электрон-позитронной аннигиляции к эффектам «новой» физики значительно усиливается при высоких энергиях, $\sqrt{s} >> 2M_W$, где важную роль играет механизм калибровочного сокращения. Дело в том, что вклад $Z^\prime$-бозона в сечение процесса нарушает механизм калибровочного сокращения, играющий важную роль в СМ. Действие механизма калибровочного сокращения состоит в том, что он обеспечивает «правильное» поведение сечения процесса электрон-позитронной аннигиляции с ростом энергии, которое не нарушает унитарный предел, несмотря на быстро растущие с энергией отдельные вклады в сечение. Вместе с тем эффекты, индуцированные появлением дополнительного калибровочного бозона, нарушают механизм калибровочного сокращения в энергетическом интервале $2M_W << \sqrt{s} << M_{Z^\prime}$, что проявляется в виде «разбалансировки» отдельных вкладов в сечение и, как следствие, в возникновении существенно иной по сравнению со СМ энергетической зависимостью сечений. Этим обусловлено действие так называемого механизма усиления эффектов «новой» физики в процессе электрон-позитронной аннигиляции. Именно в силу этого обстоятельства линейный коллайдер \textit{ILC} является одним из основных инструментариев для поиска эффектов «новой» физики при исследовании процесса электрон-позитронной аннигиляции.

Следует отметить также, что коллаборация \textit{CDF} на коллайдере тэватрон одной из первых получила прямые ограничения на угол $Z$−$Z^\prime$-смешивания из обработки данных по измерению процесса адронного рождения $W^+W^−$-бозонов. И вновь относительно небольшая энергия установки и низкая светимость не позволили улучшить ограничения, полученные на коллайдере \textit{LEP2}, а лишь повторить их.

Возможности коллайдера \textit{LНС} по обнаружению эффектов $Z$−$Z^\prime$-смешивания в процессе рождения пар заряженных калибровочных $W^±$-бозонов с их последующим распадом по чисто лептонному каналу $lvl^\prime v^\prime$. Несмотря на очевидное достоинство данного канала, связанное с подавленностью фона, особенно при больших инвариантных массах $W^±$-бозонов, у него имеется заметный недостаток, связанный с присутствием в конечных фермионных состояниях двух нейтрино, что не позволяет восстановить распределение по инвариантной массе бозонных пар из экспериментальных данных. В то же время распад пары $W^±$-бозонов по полулептонному каналу $lvjj$ свободен от указанного недостатка. В процессе $pp \rightarrow Z^\prime \rightarrow WW + X \rightarrow lvjj + X$ существует возможность реконструировать распределение по инвариантной массе $W^+W^-$- пары и тем самым исследовать резонансную структуру $Z^\prime$-бозона. Еще одним достоинством настоящего полулептонного процесса является то, что он имеет сечение, существенно превосходящее сечение чисто лептонного канала. Вместе с тем полулептонный канал, в отличие от лептонного канала $lvl^\prime v^\prime$, имеет большой КХД-фон, вызванный рождением $W_{jj}$-, а также $Z_{jj}$-состояний. В последнем случае предполагается, что $Z$-бозон распадается по лептонному каналу, а в процессе детектирования лептонов один из них теряется. Кроме перечисленных выше КХД фоновых процессов имеется еще один, который играет важную роль в оценке всей фоновой составляющей. Это процесс рождения пар $t\bar{t}$-кварков. Однако большой КХД-фон может быть редуцирован путем наложения кинематических ограничений на поперечные импульсы заряженных лептонов и адронных струй в резонансном сигнале рождения $Z^\prime$-бозонов~\cite{Bobovnikov:2016}.

